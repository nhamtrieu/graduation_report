\documentclass[../DoAn.tex]{subfiles}
\begin{document}

\begin{center}
    \Large{\textbf{TÓM TẮT NỘI DUNG ĐỒ ÁN}}\\
\end{center}
\vspace{1cm}
Từ lâu, vấn đề tai nạn giao thông do sử dụng rượu bia luôn là một mối lo ngại lớn đối với nhà nước và người dân.
Mặc dù nhà nước đã ban hành nhiều nghị định xử phạt những cá nhân vi phạm như: phạt hành chính, tước giấy phép lái xe, ... Tuy nhiên việc loại bỏ văn hóa uống rượu bia là điều không thể và không nên.
Vì vậy, để giải quyết vấn đề này, các giải pháp cho phép người dân vừa có thể sử dụng đồ uống có cồn như rượu bia mà vẫn an toàn khi tham gia giao thông đã xuất hiện như đặt xe ôm, đặt taxi hay như đặt người lái xe hộ đã xuất hiện.
Tuy nhiên, ở Việt Nam thì việc đặt xe ôm hay taxi đã khá phổ biến nhưng bên cạnh đó dịch vụ đặt người lái hộ vẫn còn khá mới mẻ và chưa được phổ biến rộng rãi.

Do đó, em quyết định xây dựng một ứng dụng cho phép mọi người có thể đặt người lái hộ khi cần thiết.
Mục tiêu của ứng dụng này là đem lại cho người dùng trải nghiệm tốt và dễ dàng để sử dụng.

Để có thể thực hiện được điều này, đồ án là sự kết hợp của framework Flutter, NestJS và Firebase.
Bên cạnh đó còn có sự hỗ trợ đến từ API của Google Map Platform hỗ trợ 
việc tìm kiếm, định tuyến các địa điểm cũng như tính toán khoảng cách và độ ưu tiên trong việc phân phối tài xế 1 cách hiệu quả.

Qua đó, ứng dụng đã phần nào giải quyết được các vấn đề ban đầu, mang lại trải nghiệm mượt mà, tiện lợi cho người dùng.
Ứng dụng không chỉ giải quyết được vấn đề tai nạn giao thông do sử dụng rượu bia mà còn góp phần nâng cao ý thức chấp hành luật an toàn giao thông của mọi người, từ đó tạo nên một môi trường giao thông văn minh, an toàn hơn.
\begin{flushright}
Sinh viên thực hiện\\
\begin{tabular}{@{}c@{}}
\textit{(Ký và ghi rõ họ tên)}
\end{tabular}
\end{flushright}

\end{document}
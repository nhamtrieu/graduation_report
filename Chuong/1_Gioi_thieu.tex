\documentclass[../DoAn.tex]{subfiles}
\begin{document}

\section{Đặt vấn đề}
\label{section:1.1}
Khi đặt vấn đề, sinh viên cần làm nổi bật mức độ cấp thiết, tầm quan trọng và/hoặc quy mô của bài toán của mình.

Gợi ý cách trình bày cho sinh viên: Xuất phát từ tình hình thực tế gì, dẫn đến vấn đề hoặc bài toán gì. Vấn đề hoặc bài toán đó, nếu được giải quyết, đem lại lợi ích gì, cho những ai, còn có thể được áp dụng vào các lĩnh vực khác nữa không. Sinh viên cần lưu ý phần này chỉ trình bày vấn đề, tuyệt đối không trình bày giải pháp.

\section{Mục tiêu và phạm vi đề tài}
\label{section:1.2}
Sinh viên trước tiên cần trình bày tổng quan các kết quả của các nghiên cứu hiện nay cho bài toán giới thiệu ở phần \ref{section:1.1} (đối với đề tài nghiên cứu), hoặc về các sản phẩm hiện tại/về nhu cầu của người dùng (đối với đề tài ứng dụng). Tiếp đến, sinh viên tiến hành so sánh và đánh giá tổng quan các sản phẩm/nghiên cứu này.

Dựa trên các phân tích và đánh giá ở trên, sinh viên khái quát lại các hạn chế hiện tại đang gặp phải. Trên cơ sở đó, sinh viên sẽ hướng tới giải quyết vấn đề cụ thể gì, khắc phục hạn chế gì, phát triển phần mềm \textbf{có các chức năng chính gì}, tạo nên đột phá gì, v.v.

Trong phần này, sinh viên lưu ý chỉ trình bày tổng quan, không đi vào chi tiết của vấn đề hoặc giải pháp. Nội dung chi tiết sẽ được trình bày trong các chương tiếp theo, đặc biệt là trong Chương 5.

\section{Định hướng giải pháp}
\label{section:1.3}
Từ việc xác định rõ nhiệm vụ cần giải quyết ở phần \ref{section:1.2}, sinh viên đề xuất định hướng giải pháp của mình theo trình tự sau: (i) Sinh viên trước tiên trình bày sẽ giải quyết vấn đề theo định hướng, phương pháp, thuật toán, kỹ thuật, hay công nghệ nào; Tiếp theo, (ii) sinh viên mô tả ngắn gọn giải pháp của mình là gì (khi đi theo định hướng/phương pháp nêu trên); và sau cùng, (iii) sinh viên trình bày đóng góp chính của đồ án là gì, kết quả đạt được là gì.

Sinh viên lưu ý không giải thích hoặc phân tích chi tiết công nghệ/thuật toán trong phần này. Sinh viên chỉ cần nêu tên định hướng công nghệ/thuật toán, mô tả ngắn gọn trong một đến hai câu và giải thích nhanh lý do lựa chọn.

\section{Bố cục đồ án}
\label{section:1.4}
Phần còn lại của báo cáo đồ án tốt nghiệp này được tổ chức như sau. 

Chương 2 trình bày về v.v. 

Trong Chương 3, em/tôi giới thiệu về v.v.

\textbf{Chú ý:} Sinh viên cần viết mô tả thành đoạn văn đầy đủ về nội dung chương. Tuyệt đối không viết ý hay gạch đầu dòng. Chương 1 không cần mô tả trong phần này. 

Ví dụ tham khảo mô tả chương trong phần bố cục đồ án tốt nghiệp: Chương *** trình bày đóng góp chính của đồ án, đó là một nền tảng ABC cho phép khai phá và tích hợp nhiều nguồn dữ liệu, trong đó mỗi nguồn dữ liệu lại có định dạng đặc thù riêng. Nền tảng ABC được phát triển dựa trên khái niệm DEF, là các module ngữ nghĩa trợ giúp người dùng tìm kiếm, tích hợp và hiển thị trực quan dữ liệu theo mô hình cộng tác và mô hình phân tán.

\textbf{Chú ý:} Trong phần nội dung chính, mỗi chương của đồ án nên có phần Tổng quan và Kết chương. Hai phần này đều có định dạng văn bản “Normal”, sinh viên không cần tạo định dạng riêng, ví dụ như không in đậm/in nghiêng, không đóng khung, v.v. 

Trong phần Tổng quan của chương N, sinh viên nên có sự liên kết với chương N-1 rồi trình bày sơ qua lý do có mặt của chương N và sự cần thiết của chương này trong đồ án. Sau đó giới thiệu những vấn đề sẽ trình bày trong chương này là gì, trong các đề mục lớn nào.

Ví dụ về phần Tổng quan: Chương 3 đã thảo luận về nguồn gốc ra đời, cơ sở lý thuyết và các nhiệm vụ chính của bài toán tích hợp dữ liệu. Chương 4 này sẽ trình bày chi tiết các công cụ tích hợp dữ liệu theo hướng tiếp cận “mashup”. Với mục đích và phạm vi của đề tài, sáu nhóm công cụ tích hợp dữ liệu chính được trình bày bao gồm: (i) nhóm công cụ ABC trong phần 4.1, (ii) nhóm công cụ DEF trong phần 4.2, nhóm công cụ GHK trong phần 4.3, v.v.

Trong phần Kết chương, sinh viên đưa ra một số kết luận quan trọng của chương. Những vấn đề mở ra trong Tổng quan cần được tóm tắt lại nội dung và cách giải quyết/thực hiện như thế nào. Sinh viên lưu ý không viết Kết chương giống hệt Tổng quan. Sau khi đọc phần Kết chương, người đọc sẽ nắm được sơ bộ nội dung và giải pháp cho các vấn đề đã trình bày trong chương. Trong Kết chương, Sinh viên nên có thêm câu liên kết tới chương tiếp theo.

Ví dụ về phần Kết chương: Chương này đã phân tích chi tiết sáu nhóm công cụ tích hợp dữ liệu. Nhóm công cụ ABC và DEF thích hợp với những bài toán tích hợp dữ liệu phạm vi nhỏ. Trong khi đó, nhóm công cụ GHK lại chứng tỏ thế mạnh của mình với những bài toán cần độ chính xác cao, v.v. Từ kết quả nghiên cứu và phân tích về sáu nhóm công cụ tích hợp dữ liệu này, tôi đã thực hiện phát triển phần mềm tự động bóc tách và tích hợp dữ liệu sử dụng nhóm công cụ GHK. Phần này được trình bày trong chương tiếp theo – Chương 5.

\end{document}
\documentclass[../DoAn.tex]{subfiles}
\begin{document}
\section{Flutter}
\label{section:3.1}

\subsection{Giới thiệu}
\label{subsection:3.1.1}
Flutter là một framework phát triển giao diện người dùng (UI)
 mã nguồn mở do Google phát triển. Nó cho phép xây dựng các ứng dụng
 đa nền tảng (cross-platform) từ một cơ sở mã duy nhất. Flutter sử dụng
 ngôn ngữ lập trình Dart và cung cấp một bộ widget phong phú để tạo 
 giao diện người dùng đẹp mắt và linh hoạt.

Flutter được sử dụng để phát triển ứng dụng di động (Android, iOS), 
ứng dụng web, và thậm chí cả ứng dụng desktop (Windows, MacOS, Linux).

\subsection{Mục đích sử dụng}
\label{subsection:3.1.2}
Trong dự án này Flutter được sử dụng để xây dựng UI cho các màn hình
của ứng dụng.

\subsection{Ưu -  Nhược điểm}
\label{subsection:3.1.3}
1. Ưu điểm
\begin{itemize}
  \item Phát triển đa nền tảng: Với mã nguồn duy nhất, Flutter cho phép phát triển ứng dụng cho nhiều nền tảng, giảm thời gian và công sức cho việc phát triển và bảo trì.
  \item Hiệu suất cao: Flutter biên dịch mã trực tiếp sang mã máy (native) thông qua Dart's Ahead-of-Time (AOT) compilation, đảm bảo hiệu suất gần như ứng dụng gốc (native).
  \item Giao diện đẹp và linh hoạt: Flutter cung cấp một bộ widget phong phú, được tùy chỉnh cao, giúp tạo ra giao diện người dùng đẹp mắt và đồng nhất trên các nền tảng.
  \item Cộng đồng phát triển mạnh mẽ: Flutter có một cộng đồng rộng lớn và tài liệu phong phú, giúp hỗ trợ giải quyết vấn đề nhanh chóng.
\end{itemize}

2. Nhược điểm
\begin{itemize}
  \item Dung lượng ứng dụng lớn: Các ứng dụng Flutter thường có dung lượng lớn hơn so với ứng dụng native, đặc biệt khi so với ứng dụng iOS.
  \item Thiếu một số thư viện native: Một số tính năng hoặc thư viện native cần phải được phát triển thêm bằng cách viết mã native (Android: Kotlin/Java, iOS: Swift/Objective-C).
  \item Hỗ trợ native không đầy đủ: Mặc dù Flutter cung cấp rất nhiều tính năng, nhưng với các ứng dụng cần tích hợp sâu vào hệ điều hành (như sử dụng Bluetooth, cảm biến), sẽ phải viết mã native thủ công.
\end{itemize}

\section{NestJS}
\label{section:3.2}
\subsection{Giới thiệu}
\label{subsection:3.2.1}
NestJS là một framework mạnh mẽ để xây dựng ứng dụng phía server (backend) trong Node.js. Nó được xây dựng trên nền tảng của TypeScript và sử dụng các concept phổ biến từ Angular như Dependency Injection, Decorators, và Modularity. NestJS kết hợp các đặc điểm tốt nhất của các framework hiện có (như Express hoặc Fastify) và cung cấp một cấu trúc ứng dụng rõ ràng, dễ mở rộng.

\subsection{Mục đích sử dụng}
\label{subsection:3.2.2}
\begin{itemize}
  \item NestJS được sử dụng để viết các API cho ứng dụng
  \item Kết hợp với Firebase, API của Google Map Platform để xây dựng các dịch vụ backend
\end{itemize}

\subsection{Ưu -  Nhược điểm}
\label{subsection:3.2.3}
1. Ưu điểm
\begin{itemize}
  \item Kiến trúc module rõ ràng, dễ mở rộng: NestJS cung cấp cấu trúc module có tổ chức, giúp phát triển và bảo trì code dễ dàng hơn.
  \item Hỗ trợ TypeScript: NestJS được xây dựng trên TypeScript, giúp phát triển ứng dụng một cách an toàn và hiệu quả hơn.
  \item Dependency Injection: NestJS sử dụng Dependency Injection để quản lý các dependencies giữa các module và services, giúp code trở nên linh hoạt và dễ dàng trong việc kiểm thử phần mềm.
  \item Cộng đồng và tài liệu phong phú: NestJS có một cộng đồng đông đảo và tài liệu chính thức chi tiết giúp lập trình viên học tập và triển khai 1 cách dễ dàng.
\end{itemize}

2. Nhược điểm
\begin{itemize}
  \item Khó tiếp cận với người mới bắt đầu: Cấu trúc module, dependency injection và các khái niệm khác của NestJS có thể gây khó khăn cho người mới bắt đầu.
  \item Kích thước bundle: Do sử dụng TypeScript và nhiều tính năng, kích thước bundle của ứng dụng NestJS có thể lớn hơn so với các ứng dụng sử dụng các framework nhẹ hơn.
\end{itemize}

\section{Firebase}
\label{section:3.3}

\subsection{Giới thiệu}
\label{subsection:3.3.1}
Firebase là một nền tảng phát triển ứng dụng di động và web do Google 
cung cấp. Nó cung cấp nhiều dịch vụ đám mây (Backend-as-a-Service) 
giúp các nhà phát triển tập trung vào việc xây dựng ứng dụng mà không 
cần quản lý cơ sở hạ tầng phía sau. Firebase cung cấp nhiều tính năng 
như xác thực người dùng, cơ sở dữ liệu thời gian thực, lưu trữ đám mây, 
phân tích và nhiều dịch vụ khác.

\subsection{Mục đích sử dụng}
\label{subsection:3.3.2}
Trong dự án này, Firebase được sử dụng để:
\begin{itemize}
  \item Realtime Database: Lưu trữ và quản lý dữ liệu thời gian thực
  \item Cloud Storage: Lưu trữ và quản lý files và hình ảnh
  \item Cloud Messaging: Gửi thông báo đẩy (push notifications) đến người dùng
\end{itemize}

\subsection{Ưu -  Nhược điểm}
\label{subsection:3.3.3}
1. Ưu điểm
\begin{itemize}
  \item Phát triển nhanh chóng: Firebase cung cấp nhiều dịch vụ được xây dựng sẵn, giúp giảm thời gian phát triển ứng dụng.
  \item Khả năng mở rộng: Firebase được xây dựng trên cơ sở hạ tầng của Google, cho phép ứng dụng dễ dàng mở rộng quy mô.
  \item Bảo mật cao: Được hỗ trợ bởi Google với các tính năng bảo mật mạnh mẽ
  \item Hỗ trợ đa nền tảng: Firebase hỗ trợ phát triển ứng dụng cho nhiều nền tảng như iOS, Android và web.
\end{itemize}

2. Nhược điểm
\begin{itemize}
  \item Giá thành cao: Firebase có giá thành cao, đặc biệt là với các dịch vụ nâng cao.
  \item Phụ thuộc vào Google: Hoàn toàn phụ thuộc vào hạ tầng của Google
  \item Tính linh hoạt hạn chế: Không thể tùy chỉnh sâu như các giải pháp backend tự xây dựng
\end{itemize}

\end{document}
